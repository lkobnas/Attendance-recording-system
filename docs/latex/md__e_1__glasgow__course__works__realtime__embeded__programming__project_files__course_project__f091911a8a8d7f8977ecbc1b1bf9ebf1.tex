在树莓派上使用\+A\+S608模块,本项目实现了所有官方用户开发手册中列出的功能,函数声明在 {\ttfamily \hyperlink{as608_8h}{as608.\+h}}中。用户可直接调用相应的函数与\+A\+S608模块进行通信。

另外,项目中有一个命令行程序,可以在终端下通过命令与模块进行交互。

\subsection*{一、\+A\+S608}

\subsubsection*{1. 简介}

数据均存储在模块内,指纹容量300枚(0-\/299)。

芯片内设有一个 72\+K 字节的图像缓冲区与二个 512 bytes(256 字)大小的特征文件 缓冲区,名字分别称为:{\ttfamily Image\+Buffer},{\ttfamily Char\+Buffer1},{\ttfamily Char\+Buffer2}。用户可以通过指 令读写任意一个缓冲区。{\ttfamily Char\+Buffer1} 或 {\ttfamily Char\+Buffer2} 既可以用于存放普通特征文件也 可以用于存放模板特征文件。

\subsubsection*{2. 工作流程}

录入指纹流程:



搜索指纹流程:



A\+S608模块内部内置了手指探测电路,用户可读取状态引脚(\+W\+A\+K)判断有无手指按下。在本项目组,{\ttfamily \hyperlink{as608_8h}{as608.\+h}}中的全局变量{\ttfamily g\+\_\+detect\+\_\+pin}就是指该引脚与树莓派的哪个\+G\+P\+I\+O端口相连的。($<$font color=\char`\"{}red\char`\"{}$>$注意:引脚编码方式是{\ttfamily wiring\+Pi}编码$<$/font$>$)。读取该引脚的输入信号,高电平意味着模块上有手指存在,否则不存在,等待几秒后,如果一直检测不到手指,就报错。

\subsubsection*{3. 芯片地址和密码}

默认地址是{\ttfamily 0xffffffff},默认密码是{\ttfamily 0x00000000},如果不自己设置其他密码,就不需要向模块验证密码,否则,与模块通信的第一条指令必须是验证密码{\ttfamily \hyperlink{as608_8c_a45a9810f3281670bff70b96a150f5c09}{P\+S\+\_\+\+Vfy\+Pwd()}}。

\subsection*{二、项目-\/函数库}

把本项目根目录下的{\ttfamily \hyperlink{as608_8h}{as608.\+h}}和{\ttfamily \hyperlink{as608_8c}{as608.\+c}}拷贝到你的程序目录下即可。

\subsubsection*{1. 模块参数变量}


\begin{DoxyCode}
1 \{C++\}
2 // typedef unsigned int uint;
3 
4 typedef struct AS608\_Module\_Info \{
5   uint status;      // 状态寄存器 0
6   uint model;       // 传感器类型 0-15
7   uint capacity;    // 指纹容量,300
8   uint secure\_level;    // 安全等级 1/2/3/4/5,默认为3
9   uint packet\_size;     // 数据包大小 32/64/128/256 bytes,默认为128
10   uint baud\_rate;       // 波特率系数 
11   uint chip\_addr;       // 设备(芯片)地址                  
12   uint password;        // 通信密码
13   char product\_sn[12];        // 产品型号
14   char software\_version[12];  // 软件版本号
15   char manufacture[12];       // 厂家名称
16   char sensor\_name[12];       // 传感器名称
17 
18   uint detect\_pin;      // AS608的WAK引脚连接的树莓派GPIO引脚号
19   uint has\_password;    // 是否有密码
20 \} AS608;
21 
22 extern AS608 g\_as608;
\end{DoxyCode}


\subsubsection*{2. 全局变量}

使用树莓派的硬件进行串口通信,需要额外配置一下(关闭板载蓝牙功能等),参考 \href{https://blog.csdn.net/guet_gjl/article/details/85164072}{\tt C\+S\+D\+N-\/树莓派利用串口进行通信}。


\begin{DoxyItemize}
\item {\ttfamily int g\+\_\+fd}:打开串口的文件描述符。
\end{DoxyItemize}


\begin{DoxyCode}
\hyperlink{as608_8c_ac09e95a250275ca69c109f7bbadfef43}{g\_fd} = serialOpen(\textcolor{stringliteral}{"/dev/ttyAMA0"}, 9600);  \textcolor{comment}{// 9600是波特率}
\end{DoxyCode}



\begin{DoxyItemize}
\item {\ttfamily int g\+\_\+verbose}:函数工作过程中输出到屏幕上信息量。为{\ttfamily 0}则显示的很少,主要是传输数据包时会显示进度条。为{\ttfamily 1}则显示详细信息,如发送的指令包内容和接收的指令包内容等。为{\ttfamily 其他}数值则不显示任何信息。
\item {\ttfamily int g\+\_\+error\+\_\+code}:模块返回的错误码 以及 自定义的错误代码。
\item {\ttfamily char g\+\_\+error\+\_\+desc\mbox{[}128\mbox{]}}:错误代码 的含义。可通过{\ttfamily char$\ast$ \hyperlink{as608_8c_af8cb71ea77ec9b32ca50d4732a6a7b67}{P\+S\+\_\+\+Get\+Error\+Desc()}}函数获得。
\end{DoxyItemize}

\subsubsection*{3. 函数}

在{\ttfamily \hyperlink{as608_8c}{as608.\+c}}中每个函数前都有详细注释。

\subsubsection*{4. 如何使用}

把本项目根目录下的{\ttfamily \hyperlink{as608_8h}{as608.\+h}}和{\ttfamily \hyperlink{as608_8c}{as608.\+c}}拷贝到你的程序目录下并包含头文件。

还需要包含 {\ttfamily $<$wiring\+Pi.\+h$>$} 和 {\ttfamily $<$wiring\+Serial.\+h$>$}。

最基础的使用如下:


\begin{DoxyCode}
\textcolor{preprocessor}{#include <stdio.h>}
\textcolor{preprocessor}{#include <wiringPi.h>}
\textcolor{preprocessor}{#include <wiringSerial.h>}
\textcolor{preprocessor}{#include "\hyperlink{as608_8h}{as608.h}"}          \textcolor{comment}{// 包含头文件}

\textcolor{comment}{// 声明全局变量【定义在as608.c】}
\textcolor{keyword}{extern} \hyperlink{struct_a_s608___module___info}{AS608} \hyperlink{as608_8c_a19c3a0b67af18ae8b8d187e21bcc6398}{g\_as608};
\textcolor{keyword}{extern} \textcolor{keywordtype}{int} \hyperlink{as608_8c_ac09e95a250275ca69c109f7bbadfef43}{g\_fd};
\textcolor{keyword}{extern} \textcolor{keywordtype}{int} \hyperlink{as608_8c_af27e3faeb692ec7ae78672182717dd18}{g\_verbose};
\textcolor{keyword}{extern} \textcolor{keywordtype}{char}  \hyperlink{as608_8c_a017d41f37a65f0f9222622b307688824}{g\_error\_desc}[];
\textcolor{keyword}{extern} \hyperlink{as608_8h_a65f85814a8290f9797005d3b28e7e5fc}{uchar} \hyperlink{as608_8c_a6957ac1dcfb11e11e72f60f78d16de53}{g\_error\_code};

\textcolor{keywordtype}{int} \hyperlink{_a_s608_2_test_2test_8cpp_a3c04138a5bfe5d72780bb7e82a18e627}{main}() \{
    \textcolor{comment}{// 给全局变量赋值}
    g\_as608.\hyperlink{struct_a_s608___module___info_aacc9cf7ae2d65e95f537eafe420f7cd0}{detect\_pin} = 1; 
    g\_as608.\hyperlink{struct_a_s608___module___info_a0fdbc2c11983db5b9007005be490be89}{has\_password} = 0;  \textcolor{comment}{// 没有密码}
    g\_verbose = 0;       \textcolor{comment}{// 显示少量输出信息}

    \textcolor{comment}{// 初始化wiringPi库}
    \textcolor{keywordflow}{if} (-1 == wiringPiSetup())
        \textcolor{keywordflow}{return} 1;

    \textcolor{comment}{// 设置g\_detect\_pin引脚为输入模式}
    pinMode(g\_as608.\hyperlink{struct_a_s608___module___info_aacc9cf7ae2d65e95f537eafe420f7cd0}{detect\_pin}, INPUT);

    \textcolor{comment}{// 打开串口}
    \textcolor{keywordflow}{if} ((g\_fd = serialOpen(\textcolor{stringliteral}{"/dev/ttyAMA0"}, 9600)) < 0)
        \textcolor{keywordflow}{return} 2;

    \textcolor{comment}{// 初始化AS608模块}
    \textcolor{keywordflow}{if} (\hyperlink{as608_8c_a522a62d36aeebd23459b2343a84a1972}{PS\_Setup}(0xfffffff, 0x00000000) == 0)
        \textcolor{keywordflow}{return} 3;

    \textcolor{comment}{/****************************************/}

    \textcolor{comment}{// do something}

    \textcolor{comment}{/****************************************/}

    \textcolor{comment}{// 关闭串口}
    serialClose(g\_fd);

    \textcolor{keywordflow}{return} 0;
\}
\end{DoxyCode}


检测手指

A\+S608使用的是电阻屏,可以通过检测{\ttfamily W\+A\+K}引脚的电平高低来判断模块上是否有手指。


\begin{DoxyCode}
\textcolor{comment}{// as608.h 中有一个封装函数}
\textcolor{comment}{// 检测到手指,返回true,否则返回false}
\textcolor{comment}{// 前提是配置了 g\_as608.detect\_pin,  即AS608的WAK引脚}
\textcolor{keywordtype}{bool} \hyperlink{as608_8c_afba869bff98f6cbdbfd50eebf101e9d1}{PS\_DetectFinger}();
\end{DoxyCode}


在{\ttfamily exanmple/main.\+c}中有两个函数


\begin{DoxyCode}
1 \{C++\}
2 // 阻塞至检测到手指,最长阻塞wait\_time毫秒
3 bool waitUntilDetectFinger(int wait\_time) \{
4     while (true) \{
5         if (PS\_DetectFinger())
6             return true;
7         else \{
8             delay(100);
9             wait\_time -= 100;
10             if (wait\_time < 0)
11                 return false;
12         \}
13     \}
14 \}
15 
16 // 阻塞至检测不到手指,最长阻塞wait\_time毫秒
17 bool waitUntilDetectFinger(int wait\_time) \{
18     while (true) \{
19         if (PS\_DetectFinger())
20             return true;
21         else \{
22             delay(100);
23             wait\_time -= 100;
24             if (wait\_time < 0)
25                 return false;
26         \}
27     \}
28 \}
\end{DoxyCode}


录入指纹


\begin{DoxyCode}
\textcolor{keywordtype}{bool} newFingerprint(\textcolor{keywordtype}{int} pageID) \{
    printf(\textcolor{stringliteral}{"Please put your finger on the module.\(\backslash\)n"});
    \textcolor{keywordflow}{if} (waitUntilDetectFinger(5000)) \{
        delay(500);
        \hyperlink{as608_8c_ab93b7b3b367d82fc358845a0713e77db}{PS\_GetImage}();
        \hyperlink{as608_8c_ac515c3ff7dce6a27509f9352cd0f1943}{PS\_GenChar}(1);
    \}
    \textcolor{keywordflow}{else} \{
        printf(\textcolor{stringliteral}{"Error: Didn't detect finger!\(\backslash\)n"});
        exit(1);
    \}

    \textcolor{comment}{// 判断用户是否抬起了手指,}
    printf(\textcolor{stringliteral}{"Ok.\(\backslash\)nPlease raise your finger!\(\backslash\)n"});
    \textcolor{keywordflow}{if} (waitUntilNotDetectFinger(5000)) \{
        delay(100);
        printf(\textcolor{stringliteral}{"Ok.\(\backslash\)nPlease put your finger again!\(\backslash\)n"});
        \textcolor{comment}{// 第二次录入指纹}
        \textcolor{keywordflow}{if} (waitUntilDetectFinger(5000)) \{
            delay(500);
            \hyperlink{as608_8c_ab93b7b3b367d82fc358845a0713e77db}{PS\_GetImage}();
            \hyperlink{as608_8c_ac515c3ff7dce6a27509f9352cd0f1943}{PS\_GenChar}(2);
        \}
        \textcolor{keywordflow}{else} \{
            printf(\textcolor{stringliteral}{"Error: Didn't detect finger!\(\backslash\)n"});
            exit(1);
        \}
    \}
    \textcolor{keywordflow}{else} \{
        printf(\textcolor{stringliteral}{"Error! Didn't raise your finger\(\backslash\)n"});
        exit(1);
    \}

    \textcolor{keywordtype}{int} score = 0;
    \textcolor{keywordflow}{if} (\hyperlink{as608_8c_aaa413f6e387308e85a872533e9c2428a}{PS\_Match}(&score)) \{
        printf(\textcolor{stringliteral}{"Matched! score=%d\(\backslash\)n"}, score);
    \}
    \textcolor{keywordflow}{else} \{
        printf(\textcolor{stringliteral}{"Not matched, raise your finger and put it on again.\(\backslash\)n"});
        exit(1);
    \}

    \textcolor{comment}{// 合并特征文件}
    \hyperlink{as608_8c_a42e7c83b9eae8089de6a4fa22b63af16}{PS\_RegModel}();
    \hyperlink{as608_8c_a23adbcbb36764d9b638bb9c7caea5b30}{PS\_StoreChar}(2, pageID);

    printf(\textcolor{stringliteral}{"OK! New fingerprint saved to pageID=%d\(\backslash\)n"}, pageID);
\}
\end{DoxyCode}


\subsection*{三、命令行程序}

\subsubsection*{1. 编译运行}


\begin{DoxyCode}
1 cd example
2 make
3 ./fp  # 第一次使用,让程序初始化
4 alias fp=./fp # 以后可以使用fp,而不用加前缀"./"
\end{DoxyCode}


\subsubsection*{2. 修改配置文件}

方法一:编辑 {\ttfamily $\sim$/.fpconfig} :执行{\ttfamily vim $\sim$/.fpconfig}


\begin{DoxyCode}
1 address=0xffffffff
2 password=none
3 baudrate=9600
4 detect\_pin=1
5 serial=/dev/ttyAMA0
\end{DoxyCode}


方法二:使用命令


\begin{DoxyItemize}
\item {\ttfamily fp cfgaddr \mbox{[}address\mbox{]}} :修改address
\item {\ttfamily fp cfgpwd \mbox{[}password\mbox{]}}:修改password
\item {\ttfamily fp cfgserial \mbox{[}serial\+File\mbox{]}}:修改串口通信端口
\item {\ttfamily fp cfgbaud \mbox{[}baudrate\mbox{]}}:修改通信波特率
\item {\ttfamily fp cfgpin \mbox{[}G\+P\+I\+O\+\_\+pin\mbox{]}}:修改检测手指是否存在 对于的\+G\+P\+I\+O引脚
\end{DoxyItemize}

\subsubsection*{3. 如何使用}

{\ttfamily fp -\/h} :显示使用帮助


\begin{DoxyCode}
A command line program to interact with \hyperlink{struct_a_s608___module___info}{AS608} module.

Usage:
  ./fp [command] [param] [option]

Available Commands:
-------------------------------------------------------------------------
  command  | param     | description
-------------------------------------------------------------------------
  cfgaddr   [addr]     \hyperlink{struct___config}{Config} address in local config file
  cfgpwd    [pwd]      \hyperlink{struct___config}{Config} password in local config file
  cfgserial [serialFile] \hyperlink{struct___config}{Config} serial port in local config file. Default:/dev/ttyAMA0
  cfgbaud   [rate]     \hyperlink{struct___config}{Config} baud rate in local config file
  cfgpin    [GPIO\_pin] \hyperlink{struct___config}{Config} GPIO pin to detect finger in local confilg file

  add       [pID]      Add a \textcolor{keyword}{new} fingerprint to database. (Read twice) 
  enroll    []         Add a \textcolor{keyword}{new} fingerprint to database. (Read only once)
  \textcolor{keyword}{delete}    [pID \{count\}]  Delete one or contiguous fingerprints.
  empty     []         Empty the database.
  search    []         Collect fingerprint and search in database.
  identify  []         Search
  count     []         Get the count of registered fingerprints.
  list      []         Show the registered fingerprints list.
  info      []         Show the basic parameters of the module.
  random    []         Generate a random number.(0~2^32)

  getimage  []         Collect a fingerprint and store to ImageBuffer.
  upimage   [filename] Download finger image to ras-pi in ImageBuffer of the module
  downimage [filename] Upload finger image to module
  genchar   [cID]      Generate fingerprint feature from ImageBuffer.
  match     []         Accurate comparison of CharBuffer1 and CharBuffer2
                         feature files.
  regmodel  []         \hyperlink{as608_8c_afe3350364056dcef1a4b3cafb9af7150}{Merge} the characteristic file in CharBuffer1 and
                         CharBuffer2 and then generate the \textcolor{keyword}{template}, the
                         results are stored in CharBuffer1 and CharBuffer2.
  storechar [cID pID]  Save the \textcolor{keyword}{template} file in CharBuffer1 or CharBuffer2
                         to the flash database location with the PageID number
  loadchar  [cID pID]  Reads the fingerprint \textcolor{keyword}{template} with the ID specified
                         in the flash database into the \textcolor{keyword}{template} buffer,
                         CharBuffer1 or CharBuffer2
  readinf   [filename] Read the FLASH Info Page (512bytes), and save to file
  writenote     [page \{note\}]   Write note loacted in pageID=page
  readnote      [page]          Read note loacted in pageID=page
  upchar        [cID filename]  Download feature file in CharBufferID to ras-pi
  downchar      [cID filename]  Upload feature file in loacl disk to module
  setpwd        [pwd]           Set password
  vfypwd        [pwd]           Verify password
  packetsize    [\{size\}]        Show or Set data packet size
  baudrate      [\{rate\}]        Show or Set baud rate
  level         [\{level\}]       Show or Set secure level(1~5)
  address       [\{addr\}]        Show or Set secure level(1~5)

Avaiable options:
  -h    Show help
  -v    Shwo details while excute the order

Usage:
  ./fp [command] [param] [option]
\end{DoxyCode}


{\bfseries 注意事项}


\begin{DoxyItemize}
\item 选项 {\ttfamily -\/v} 或 {\ttfamily -\/h} $<$font color=\char`\"{}red\char`\"{}$>$必须写到最后面$<$/font$>$,否则可能出错
\item {\ttfamily \mbox{[}\mbox{]}}中为命令对应的参数,{\ttfamily \{\}}中的表示可选。
\end{DoxyItemize}

\subsubsection*{4. 示例}


\begin{DoxyCode}
1 # 录指纹(采集两次),保存到指纹库的第7号位置
2 fp add 7
3 # 录指纹(采集一次),返回保存的位置id号
4 fp enroll
5 
6 # 删除指纹库中第5号指纹
7 fp delete 5
8 # 删除指纹库中第0号至第19号(共20个)
9 fp delete 0 20
10 
11 # 采集并比对指纹,以下3条均可
12 fp search
13 fp hsearch  # high speed search
14 fp identity
15 
16 # 列出指纹库中的指纹ID
17 fp list
18 
19 # 显示当前的芯片地址
20 fp address
21 # 设置芯片地址为0xefefefef
22 fp address 0xefefefef  # 前缀0x可省略
23 
24 # 设置密码为0xcc0825cc
25 fp setpwd 0xcc0825cc
\end{DoxyCode}


【以下图片以实际执行输出为准,可能有差别之处】





\subsection*{E\+N\+D}

\href{mailto:leopard.c@outlook.com}{\tt leopard.\+c@outlook.\+com} 